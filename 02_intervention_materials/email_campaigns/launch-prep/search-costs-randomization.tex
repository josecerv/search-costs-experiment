% Options for packages loaded elsewhere
\PassOptionsToPackage{unicode}{hyperref}
\PassOptionsToPackage{hyphens}{url}
%
\documentclass[
]{article}
\usepackage{amsmath,amssymb}
\usepackage{iftex}
\ifPDFTeX
  \usepackage[T1]{fontenc}
  \usepackage[utf8]{inputenc}
  \usepackage{textcomp} % provide euro and other symbols
\else % if luatex or xetex
  \usepackage{unicode-math} % this also loads fontspec
  \defaultfontfeatures{Scale=MatchLowercase}
  \defaultfontfeatures[\rmfamily]{Ligatures=TeX,Scale=1}
\fi
\usepackage{lmodern}
\ifPDFTeX\else
  % xetex/luatex font selection
\fi
% Use upquote if available, for straight quotes in verbatim environments
\IfFileExists{upquote.sty}{\usepackage{upquote}}{}
\IfFileExists{microtype.sty}{% use microtype if available
  \usepackage[]{microtype}
  \UseMicrotypeSet[protrusion]{basicmath} % disable protrusion for tt fonts
}{}
\makeatletter
\@ifundefined{KOMAClassName}{% if non-KOMA class
  \IfFileExists{parskip.sty}{%
    \usepackage{parskip}
  }{% else
    \setlength{\parindent}{0pt}
    \setlength{\parskip}{6pt plus 2pt minus 1pt}}
}{% if KOMA class
  \KOMAoptions{parskip=half}}
\makeatother
\usepackage{xcolor}
\usepackage[margin=1in]{geometry}
\usepackage{color}
\usepackage{fancyvrb}
\newcommand{\VerbBar}{|}
\newcommand{\VERB}{\Verb[commandchars=\\\{\}]}
\DefineVerbatimEnvironment{Highlighting}{Verbatim}{commandchars=\\\{\}}
% Add ',fontsize=\small' for more characters per line
\usepackage{framed}
\definecolor{shadecolor}{RGB}{248,248,248}
\newenvironment{Shaded}{\begin{snugshade}}{\end{snugshade}}
\newcommand{\AlertTok}[1]{\textcolor[rgb]{0.94,0.16,0.16}{#1}}
\newcommand{\AnnotationTok}[1]{\textcolor[rgb]{0.56,0.35,0.01}{\textbf{\textit{#1}}}}
\newcommand{\AttributeTok}[1]{\textcolor[rgb]{0.13,0.29,0.53}{#1}}
\newcommand{\BaseNTok}[1]{\textcolor[rgb]{0.00,0.00,0.81}{#1}}
\newcommand{\BuiltInTok}[1]{#1}
\newcommand{\CharTok}[1]{\textcolor[rgb]{0.31,0.60,0.02}{#1}}
\newcommand{\CommentTok}[1]{\textcolor[rgb]{0.56,0.35,0.01}{\textit{#1}}}
\newcommand{\CommentVarTok}[1]{\textcolor[rgb]{0.56,0.35,0.01}{\textbf{\textit{#1}}}}
\newcommand{\ConstantTok}[1]{\textcolor[rgb]{0.56,0.35,0.01}{#1}}
\newcommand{\ControlFlowTok}[1]{\textcolor[rgb]{0.13,0.29,0.53}{\textbf{#1}}}
\newcommand{\DataTypeTok}[1]{\textcolor[rgb]{0.13,0.29,0.53}{#1}}
\newcommand{\DecValTok}[1]{\textcolor[rgb]{0.00,0.00,0.81}{#1}}
\newcommand{\DocumentationTok}[1]{\textcolor[rgb]{0.56,0.35,0.01}{\textbf{\textit{#1}}}}
\newcommand{\ErrorTok}[1]{\textcolor[rgb]{0.64,0.00,0.00}{\textbf{#1}}}
\newcommand{\ExtensionTok}[1]{#1}
\newcommand{\FloatTok}[1]{\textcolor[rgb]{0.00,0.00,0.81}{#1}}
\newcommand{\FunctionTok}[1]{\textcolor[rgb]{0.13,0.29,0.53}{\textbf{#1}}}
\newcommand{\ImportTok}[1]{#1}
\newcommand{\InformationTok}[1]{\textcolor[rgb]{0.56,0.35,0.01}{\textbf{\textit{#1}}}}
\newcommand{\KeywordTok}[1]{\textcolor[rgb]{0.13,0.29,0.53}{\textbf{#1}}}
\newcommand{\NormalTok}[1]{#1}
\newcommand{\OperatorTok}[1]{\textcolor[rgb]{0.81,0.36,0.00}{\textbf{#1}}}
\newcommand{\OtherTok}[1]{\textcolor[rgb]{0.56,0.35,0.01}{#1}}
\newcommand{\PreprocessorTok}[1]{\textcolor[rgb]{0.56,0.35,0.01}{\textit{#1}}}
\newcommand{\RegionMarkerTok}[1]{#1}
\newcommand{\SpecialCharTok}[1]{\textcolor[rgb]{0.81,0.36,0.00}{\textbf{#1}}}
\newcommand{\SpecialStringTok}[1]{\textcolor[rgb]{0.31,0.60,0.02}{#1}}
\newcommand{\StringTok}[1]{\textcolor[rgb]{0.31,0.60,0.02}{#1}}
\newcommand{\VariableTok}[1]{\textcolor[rgb]{0.00,0.00,0.00}{#1}}
\newcommand{\VerbatimStringTok}[1]{\textcolor[rgb]{0.31,0.60,0.02}{#1}}
\newcommand{\WarningTok}[1]{\textcolor[rgb]{0.56,0.35,0.01}{\textbf{\textit{#1}}}}
\usepackage{graphicx}
\makeatletter
\def\maxwidth{\ifdim\Gin@nat@width>\linewidth\linewidth\else\Gin@nat@width\fi}
\def\maxheight{\ifdim\Gin@nat@height>\textheight\textheight\else\Gin@nat@height\fi}
\makeatother
% Scale images if necessary, so that they will not overflow the page
% margins by default, and it is still possible to overwrite the defaults
% using explicit options in \includegraphics[width, height, ...]{}
\setkeys{Gin}{width=\maxwidth,height=\maxheight,keepaspectratio}
% Set default figure placement to htbp
\makeatletter
\def\fps@figure{htbp}
\makeatother
\setlength{\emergencystretch}{3em} % prevent overfull lines
\providecommand{\tightlist}{%
  \setlength{\itemsep}{0pt}\setlength{\parskip}{0pt}}
\setcounter{secnumdepth}{-\maxdimen} % remove section numbering
\renewcommand{\contentsname}{Items}
\ifLuaTeX
  \usepackage{selnolig}  % disable illegal ligatures
\fi
\usepackage{bookmark}
\IfFileExists{xurl.sty}{\usepackage{xurl}}{} % add URL line breaks if available
\urlstyle{same}
\hypersetup{
  pdftitle={Search Costs Email Randomization},
  hidelinks,
  pdfcreator={LaTeX via pandoc}}

\title{Search Costs Email Randomization}
\author{}
\date{\vspace{-2.5em}June 22, 2024}

\begin{document}
\maketitle

{
\setcounter{tocdepth}{3}
\tableofcontents
}
\newpage

\subsection{calculate seminar bins based on number of seminars per
department}\label{calculate-seminar-bins-based-on-number-of-seminars-per-department}

\begin{Shaded}
\begin{Highlighting}[]
\CommentTok{\# Count the number of seminars per department}
\NormalTok{seminar\_counts }\OtherTok{\textless{}{-}}\NormalTok{ data }\SpecialCharTok{\%\textgreater{}\%}
  \FunctionTok{group\_by}\NormalTok{(department) }\SpecialCharTok{\%\textgreater{}\%}
  \FunctionTok{summarize}\NormalTok{(}\AttributeTok{seminar\_count =} \FunctionTok{n}\NormalTok{()) }\SpecialCharTok{\%\textgreater{}\%}
  \FunctionTok{arrange}\NormalTok{(seminar\_count)}

\CommentTok{\# Define bins based on the seminar counts by department}
\NormalTok{bins }\OtherTok{\textless{}{-}} \FunctionTok{cut}\NormalTok{(seminar\_counts}\SpecialCharTok{$}\NormalTok{seminar\_count, }
            \AttributeTok{breaks =} \FunctionTok{c}\NormalTok{(}\DecValTok{0}\NormalTok{, }\DecValTok{1}\NormalTok{, }\DecValTok{3}\NormalTok{, }\DecValTok{5}\NormalTok{,}\DecValTok{7}\NormalTok{,}\DecValTok{11}\NormalTok{,}\DecValTok{17}\NormalTok{, }\DecValTok{26}\NormalTok{),}
            \AttributeTok{include.lowest =} \ConstantTok{TRUE}\NormalTok{,}
            \AttributeTok{right =} \ConstantTok{TRUE}\NormalTok{)}

\CommentTok{\# Add bin information to the seminar counts}
\NormalTok{seminar\_counts }\OtherTok{\textless{}{-}}\NormalTok{ seminar\_counts }\SpecialCharTok{\%\textgreater{}\%}
  \FunctionTok{mutate}\NormalTok{(}\AttributeTok{bin\_category =}\NormalTok{ bins)}

\CommentTok{\# Summarize and print bins}
\NormalTok{bin\_summary }\OtherTok{\textless{}{-}}\NormalTok{ seminar\_counts }\SpecialCharTok{\%\textgreater{}\%}
  \FunctionTok{group\_by}\NormalTok{(bin\_category) }\SpecialCharTok{\%\textgreater{}\%}
  \FunctionTok{summarize}\NormalTok{(}\AttributeTok{department\_count =} \FunctionTok{n}\NormalTok{(),}
            \AttributeTok{total\_seminars =} \FunctionTok{sum}\NormalTok{(seminar\_count))}

\CommentTok{\# Print the summary}
\FunctionTok{print}\NormalTok{(bin\_summary)}
\end{Highlighting}
\end{Shaded}

\begin{verbatim}
## # A tibble: 7 x 3
##   bin_category department_count total_seminars
##   <fct>                   <int>          <int>
## 1 [0,1]                     279            279
## 2 (1,3]                     117            281
## 3 (3,5]                      68            300
## 4 (5,7]                      43            278
## 5 (7,11]                     32            298
## 6 (11,17]                    23            323
## 7 (17,26]                     6            128
\end{verbatim}

\newpage

\subsection{randomization within each
bin}\label{randomization-within-each-bin}

\begin{Shaded}
\begin{Highlighting}[]
\CommentTok{\# Set seed for reproducibility}
\FunctionTok{set.seed}\NormalTok{(}\DecValTok{114}\NormalTok{)}

\CommentTok{\# Function to perform stratified randomization}
\NormalTok{stratified\_randomize }\OtherTok{\textless{}{-}} \ControlFlowTok{function}\NormalTok{(data, strata\_col, group\_col, num\_groups) \{}
\NormalTok{  data }\SpecialCharTok{\%\textgreater{}\%}
    \FunctionTok{group\_by}\NormalTok{(}\FunctionTok{across}\NormalTok{(}\FunctionTok{all\_of}\NormalTok{(strata\_col))) }\SpecialCharTok{\%\textgreater{}\%}
    \FunctionTok{mutate}\NormalTok{(}
\NormalTok{      \{\{group\_col\}\} }\SpecialCharTok{:=} \FunctionTok{sample}\NormalTok{(}\FunctionTok{rep}\NormalTok{(}\FunctionTok{c}\NormalTok{(}\StringTok{"control"}\NormalTok{, }\StringTok{"treatment"}\NormalTok{), }\AttributeTok{each =} \FunctionTok{ceiling}\NormalTok{(}\FunctionTok{n}\NormalTok{() }\SpecialCharTok{/}\NormalTok{ num\_groups), }\AttributeTok{length.out =} \FunctionTok{n}\NormalTok{()))}
\NormalTok{    ) }\SpecialCharTok{\%\textgreater{}\%}
    \FunctionTok{ungroup}\NormalTok{()}
\NormalTok{\}}

\CommentTok{\# Define number of groups}
\NormalTok{num\_groups }\OtherTok{\textless{}{-}} \DecValTok{2}

\CommentTok{\# Apply stratified randomization}
\NormalTok{randomized\_data }\OtherTok{\textless{}{-}} \FunctionTok{stratified\_randomize}\NormalTok{(seminar\_counts, }\StringTok{"bin\_category"}\NormalTok{, }\StringTok{"condition"}\NormalTok{, num\_groups)}

\CommentTok{\# Check the resulting distribution}
\NormalTok{randomized\_distribution }\OtherTok{\textless{}{-}}\NormalTok{ randomized\_data }\SpecialCharTok{\%\textgreater{}\%}
  \FunctionTok{group\_by}\NormalTok{(bin\_category, condition) }\SpecialCharTok{\%\textgreater{}\%}
  \FunctionTok{summarize}\NormalTok{(}\AttributeTok{department\_count =} \FunctionTok{n}\NormalTok{(), }\AttributeTok{total\_seminars =} \FunctionTok{sum}\NormalTok{(seminar\_count), }\AttributeTok{.groups =} \StringTok{\textquotesingle{}drop\textquotesingle{}}\NormalTok{)}

\FunctionTok{print}\NormalTok{(randomized\_distribution)}
\end{Highlighting}
\end{Shaded}

\begin{verbatim}
## # A tibble: 14 x 4
##    bin_category condition department_count total_seminars
##    <fct>        <chr>                <int>          <int>
##  1 [0,1]        control                140            140
##  2 [0,1]        treatment              139            139
##  3 (1,3]        control                 59            137
##  4 (1,3]        treatment               58            144
##  5 (3,5]        control                 34            152
##  6 (3,5]        treatment               34            148
##  7 (5,7]        control                 22            141
##  8 (5,7]        treatment               21            137
##  9 (7,11]       control                 16            146
## 10 (7,11]       treatment               16            152
## 11 (11,17]      control                 12            175
## 12 (11,17]      treatment               11            148
## 13 (17,26]      control                  3             60
## 14 (17,26]      treatment                3             68
\end{verbatim}

\newpage

\subsection{chisq tests measuring whether randomization
worked}\label{chisq-tests-measuring-whether-randomization-worked}

\subsubsection{seminars in each
discipline}\label{seminars-in-each-discipline}

\begin{verbatim}
##            discipline
## condition   Chemistry Computer Science Mathematics Mechanical Engineering
##   control         136               84         481                     51
##   treatment       162               87         438                     49
##            discipline
## condition   Physics
##   control       199
##   treatment     200
\end{verbatim}

\begin{verbatim}
## 
##  Pearson's Chi-squared test
## 
## data:  table(merged_data$discipline, merged_data$condition)
## X-squared = 4.2566, df = 4, p-value = 0.3724
\end{verbatim}

\newpage

\subsubsection{seminars in each department
bin}\label{seminars-in-each-department-bin}

\begin{verbatim}
## # A tibble: 14 x 2
##    department_count condition
##               <int> <chr>    
##  1              140 control  
##  2              139 treatment
##  3               59 control  
##  4               58 treatment
##  5               34 control  
##  6               34 treatment
##  7               22 control  
##  8               21 treatment
##  9               16 control  
## 10               16 treatment
## 11               12 control  
## 12               11 treatment
## 13                3 control  
## 14                3 treatment
\end{verbatim}

\begin{verbatim}
## 
##  Pearson's Chi-squared test
## 
## data:  table(randomized_distribution$department_count, randomized_distribution$condition)
## X-squared = 8, df = 10, p-value = 0.6288
\end{verbatim}

\begin{verbatim}
## "x"
## "randomized_data.csv"
\end{verbatim}

\newpage

\subsubsection{seminars in each bin}\label{seminars-in-each-bin}

\begin{verbatim}
## 
##  Pearson's Chi-squared test
## 
## data:  table(department, condition)
## X-squared = 279, df = 278, p-value = 0.4718
## 
## 
##  Pearson's Chi-squared test
## 
## data:  table(department, condition)
## X-squared = 117, df = 116, p-value = 0.4565
## 
## 
##  Pearson's Chi-squared test
## 
## data:  table(department, condition)
## X-squared = 68, df = 67, p-value = 0.4429
## 
## 
##  Pearson's Chi-squared test
## 
## data:  table(department, condition)
## X-squared = 43, df = 42, p-value = 0.4282
## 
## 
##  Pearson's Chi-squared test
## 
## data:  table(department, condition)
## X-squared = 32, df = 31, p-value = 0.4167
## 
## 
##  Pearson's Chi-squared test
## 
## data:  table(department, condition)
## X-squared = 23, df = 22, p-value = 0.4017
## 
## 
##  Pearson's Chi-squared test
## 
## data:  table(department, condition)
## X-squared = 6, df = 5, p-value = 0.3062
\end{verbatim}

\begin{verbatim}
## # A tibble: 2 x 2
##   condition  mean
##   <chr>     <dbl>
## 1 control    3.33
## 2 treatment  3.32
\end{verbatim}

\end{document}
